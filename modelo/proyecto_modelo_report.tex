\documentclass{EFUCVproyect}
\usepackage{titlesec}
%\usepackage{pst-barcode}
%\usepackage{auto-pst-pdf} % uncomment this if used with pdflatex

\usepackage{hyperref}
\usepackage{qrcode}

\titleformat{\chapter}
  {\Large\bfseries} % format
  {}                % label
  {0pt}             % sep
  {\huge}           % before-code
\newcommand{\cb}[1]{
\medskip 
\begin{tabular}{p{5.0cm} p{9.0cm}}
& \\ 
&
\begin{tabular}{|p{9.0cm}|}
\hline
\\
#1\\
\\
\hline
\end{tabular}
\\
& \\
\end{tabular}
\medskip 
}
\titulo{{\bfseries\uppercase{\expandafter{ proyecto modelo (título del proyecto)}}}}
\autor{Nomen{ } Nescio}{m}{nomen.nescio@ucv.ve} %nombre del autor, género m o f, email
\centro{centro}
\laboratorio{laboratorio}
\title{proyecto modelo}
\author{Nomen~ Nescio}
\numproy{EF-PGXXX001-2016 }

\anho{Marzo-2015}
\begin{document}
%\maketitle
\pagestyle{empty}
%\portada %fabrica la portada usar, solo se imprime en versión final
%
%\cleardoublepage
%\primerapagina %fabrica la primera página (es igual a la portada, sin el marco)
%
%\cleardoublepage

\begin{tabular}{p{11.0cm} p{3.0cm}}
\begin{tabular}{p{11.0cm}}
\tesistitulo \\
\tesisautor \\
\tesisautoremail \\
{\sc universidad central de venezuela} \\
{\sc facultad de ciencias escuela de física} \\
{\sc \lab} \\
{\sc \centrolab} \\
\numerodelproyecto
\end{tabular}
&
\begin{tabular}{p{3.0cm}}

    \qrcode{\numerodelproyecto \\ \tesisautor \\ \tesisautoremail}

\end{tabular}
\\
& \\
\end{tabular}

\tableofcontents
\newpage
\pagestyle{fancy}
%\thispagestyle{empty}
\section{Resumen}
Lorem ipsum dolor sit amet, consectetur adipiscing elit. Ut in eros ante. Vivamus sit amet nibh quam. Sed maximus metus leo, at blandit nisi eleifend at. Aliquam nisi purus, tempor eu mi ut, ultrices tincidunt quam. Etiam mollis eros ac est feugiat fermentum quis nec sem. Nunc id malesuada neque, non fringilla sapien. Lorem ipsum dolor sit amet, consectetur adipiscing elit. Aliquam erat volutpat. Nam mattis semper nulla, dictum malesuada eros suscipit sit amet. Vivamus commodo ante ornare ex consequat, ac finibus erat accumsan. Praesent quis nulla nec ipsum gravida fermentum et eu dolor. Ut in nibh convallis, elementum urna. {\bf \Large Hasta 100 palabras}
%\chapter{Proyecto}
\section{Introducción}
%
Suspendisse non hendrerit nisl, ut suscipit est \cite{Ade:2014xna}. Fusce posuere imperdiet orci vel interdum. Praesent malesuada nec metus eget pulvinar. Duis a ultricies justo. Suspendisse ullamcorper, nibh vel semper rhoncus, nulla massa gravida metus, et tincidunt erat massa et nisi. Aenean aliquet libero fermentum lorem aliquam viverra \cite{blackholes}. Proin turpis ante, fringilla a feugiat non, maximus sed arcu. Vivamus porttitor, magna fringilla gravida dapibus, enim augue porta libero, ut congue odio arcu id nulla. Aenean ex diam, ultricies vel tincidunt sit amet, vestibulum quis ante. Sed pulvinar bibendum massa sed laoreet. Etiam vehicula velit eget elementum tempus. Fusce dignissim, velit quis sodales pharetra, massa justo sagittis turpis, id rhoncus lacus ante eget nisi. Ut efficitur mi ex.

Nulla fringilla mi ac metus fringilla pellentesque. Donec euismod sapien nec justo pellentesque, in blandit eros facilisis. Etiam in tellus ornare, fermentum neque non, dapibus lorem. Nam interdum vulputate magna, vel vulputate sem ultrices vel. Maecenas fringilla ac ante at finibus. Vivamus luctus felis a nibh viverra scelerisque. Suspendisse eu pulvinar urna. Mauris maximus in felis nec elementum. Vivamus eros ligula, aliquet et ornare eget, convallis nec massa. Aliquam nulla turpis, dignissim ac sagittis vitae, scelerisque quis eros. Praesent id lectus quam. Ut rhoncus libero. {\bf \Large Hasta 200 palabras}
\section{Justificación}
%
Vestibulum ut sapien nunc. Sed vel dui quis tellus vestibulum rutrum in vitae ipsum. Cum sociis natoque penatibus et magnis dis parturient montes, nascetur ridiculus mus. Praesent rutrum orci metus, ut consectetur libero malesuada sit amet. Nullam nec iaculis dui. Fusce vel aliquam nulla. Aliquam imperdiet erat quis nibh facilisis, quis rhoncus lacus sagittis. Nulla ornare viverra lobortis. Suspendisse potenti. Mauris iaculis nulla eu eros sollicitudin lobortis \cite{Gates:1985bt}. 
Aenean bibendum elementum ex at vulputate \cite{Gates:1985gk}. Proin sit amet mi urna. Proin non viverra sem, nec volutpat mauris. Proin vestibulum nunc id sem elementum fringilla. Donec pretium, felis nec mattis convallis, odio urna semper nisl, vel dignissim metus sapien eu nisl. In hac habitasse platea dictumst. In hac habitasse platea dictumst. Mauris id convallis lorem. Nam ut feugiat justo. Ut a diam tristique, faucibus sapien eu, malesuada urna. Fusce ornare eget felis sagittis pretium. Vestibulum ornare, urna non iaculis fringilla, quam tortor varius dui, ac imperdiet dui neque quis elit. Praesent dignissim ut arcu eu placerat. Aliquam quis consectetur nulla \cite{book:1351506}.
Maecenas consequat augue lacus, non congue turpis malesuada eu. Maecenas facilisis dui viverra nibh volutpat porttitor. Quisque ac dui nisl. Nam vitae sem sit amet eros hendrerit dictum id ut neque. Cum sociis natoque penatibus et magnis dis parturient montes, nascetur ridiculus mus. Pellentesque tincidunt blandit nunc, a tempor nulla condimentum in. Integer sed congue massa. Praesent ornare risus non pharetra pharetra. Pellentesque egestas felis quis maximus sollicitudin. Nulla fringilla libero quis felis volutpat, eu cursus ante mollis. Aliquam id pretium ligula \cite{book:1351506}.
Pellentesque condimentum a sapien ac accumsan. In gravida enim ut diam ornare luctus. Aenean placerat velit a nulla rutrum, in malesuada dui suscipit. Nulla porta enim odio, efficitur feugiat mauris porta ac. Curabitur nunc quam, pulvinar non neque in, facilisis tristique erat. Mauris sit amet tortor sem. Curabitur faucibus fermentum velit vestibulum viverra. Vivamus imperdiet sapien sed ligula placerat consequat a in arcu. Integer vestibulum dignissim est, eget eleifend augue consequat in. Phasellus lectus tellus, laoreet eget dolor in, mattis dignissim urna. Quisque lacinia tincidunt imperdiet. Maecenas vel arcu id erat semper porta sit amet in sem. Fusce eu cursus nibh.
Ut cursus, augue nec dictum fringilla, mauris orci imperdiet diam, at tempus odio lorem a metus. Praesent dapibus dolor ac blandit vestibulum. Donec vehicula in risus at blandit. Ut pretium elementum risus id pretium. Lorem ipsum dolor sit amet, consectetur adipiscing elit. {\bf \Large Hasta 400 palabras}
\section{Objetivos}
%
\subsection{Objetivo principal}
%
Sed varius libero mi, iaculis faucibus erat pharetra vel. Sed volutpat ultricies arcu, et consectetur mi venenatis tempor. Duis et ornare lorem. Morbi varius varius pretium. Nunc et ante a massa viverra aliquet eu et ante. Nullam nec nulla quis nisi interdum rutrum at quis mauris. 
%
\subsection{Objetivos espec\'ificos}
\begin{itemize}
\item Ut non diam neque. Ut dapibus velit at mauris varius, nec volutpat nisl bibendum. Proin egestas enim et diam mattis ullamcorper.
%
\item Ut commodo enim quis auctor lobortis. Curabitur iaculis nunc tempor ornare elementum. Nullam a semper turpis, eget interdum tellus.
%
\item Class aptent taciti sociosqu ad litora torquent per conubia nostra, per inceptos himenaeos. Suspendisse consectetur mattis purus eu rhoncus. Nullam non rutrum arcu.
\end{itemize}

\section{Área}
\subsection{Área de conocimiento general}
{\tiny\color{red} Indique el área general (o las áreas generales) donde se clasifiquen los resultados del proyecto, siguiendo la clasificación de la UCV: Ambiente, Salud, Energía o Ciencia Básica}

{\small
\begin{tabular}{| p{2.5cm} | p{2.5cm} | p{2.5cm} | p{2.5cm} | p{2.5cm} |}
\hline 
 & \begin{tabular}{p{2.5cm}} Ambiente \end{tabular} & \begin{tabular}{p{2.5cm}} Salud \end{tabular}  & \begin{tabular}{p{2.5cm}} Energía \end{tabular}  & \begin{tabular}{p{2.5cm}} Ciencia \\ Básica \end{tabular} \\ 
\hline
\begin{tabular}{p{2.5cm}} Objetivo 1 \end{tabular} & & & & \bf X \\  \hline
\begin{tabular}{p{2.5cm}} Objetivo 2 \end{tabular} & &\bf X & & \\  \hline
\begin{tabular}{p{2.5cm}} Objetivo 3\end{tabular} & & & & \bf X \\  \hline
\end{tabular}
}

\newpage
\subsection{Área específica}
{\tiny \color{red} Indique el área específica en las lineas de investigación de la Escuela. Son tres dimensiones: área básica en horizontal, área de aplicación en vertical y metodología indicada por las características: A,B,C ...}

{\small
\begin{tabular}{| p{2.5cm} | p{2.5cm} | p{2.5cm} | p{2.5cm} | p{2.5cm} |}
\hline 
 & \begin{tabular}{p{2.5cm}} Interacciones \\ y estructura \\ de la materia \\ a escala \\ fundamental \end{tabular} & \begin{tabular}{p{2.5cm}} Materia \\ Condensada y \\ Física de los \\ Materiales \end{tabular}  & \begin{tabular}{p{2.5cm}} Física \\ Estadística y \\ Sistemas \\ Complejos \end{tabular}  & \begin{tabular}{p{2.5cm}} Gravitación y \\ Cosmología \end{tabular} \\ 
\hline
\begin{tabular}{p{2.5cm}} Relatividad \\ General y \\ Astrofísica \end{tabular} & & & & \bf A, D, E \\  \hline
\begin{tabular}{p{2.5cm}} Física Médica \end{tabular} & &\bf C, E & & \\  \hline
\begin{tabular}{p{2.5cm}} Teoría de \\ Campos, \\ Física de \\ Partículas \\ y Teoría de \\ Cuerdas \end{tabular} & & & & \bf A, D, E \\  \hline
\begin{tabular}{p{2.5cm}} Física \\ Molecular \end{tabular} & & & & \\  \hline
\begin{tabular}{p{2.5cm}} Física de \\ Superficies \end{tabular} & & & & \\  \hline
\begin{tabular}{p{2.5cm}} Fenómenos no \\ Lineales \end{tabular} & & & & \\  \hline
\begin{tabular}{p{2.5cm}} Geofísica \end{tabular} & & & & \\  \hline
\begin{tabular}{p{2.5cm}} Mecánica \\ Cuántica \end{tabular} & & & & \\  \hline
\begin{tabular}{p{2.5cm}} Magnetismo \\ en la Materia \end{tabular} & & & & \\  \hline
\begin{tabular}{p{2.5cm}} Pa\-ra\-mag\-ne\-tis\-mo \\ en la Materia \end{tabular} & & & & \\  \hline
\begin{tabular}{p{2.5cm}} Sistemas \\ Mesoscópicos \\ y  Nanociencia \end{tabular} & & & & \\  \hline
\end{tabular}
}
\begin{description}
\item[A] Medición o caracterización de propiedades fundamentales.
\item[B] Síntesis de sistemas físicos.
\item[C] Desarrollo analítico de modelos teóricos.
\item[D] Instrumentación de equipos de medición y control.
\item[E] Desarrollo y aplicación de Métodos Computacionales.
\end{description}
\section{Metodología}
%
Vivamus scelerisque eu magna quis cursus. Vestibulum id libero euismod, pulvinar tortor ac, tincidunt mi. In ultrices vulputate diam, quis dignissim ante posuere at. Suspendisse malesuada odio tellus, id elementum tortor rutrum vitae. Donec sollicitudin placerat laoreet. Donec tempus nibh eget congue ultrices. Donec at volutpat turpis. Vivamus vel mi id libero sagittis rhoncus. Donec porta sem ipsum, et dignissim metus vulputate vel. Cras varius convallis odio, eu tristique nisl euismod ut. Etiam varius sem id velit scelerisque, quis sodales ex convallis. Donec egestas luctus pharetra. Etiam metus erat, volutpat eu interdum a, efficitur at lacus. Vestibulum ultricies tortor in urna mollis condimentum. Morbi non augue consequat, malesuada nisl quis, varius risus. Sed pretium vulputate libero sed maximus.
Nam id malesuada magna. Nam mattis porta ligula, vitae aliquam arcu vulputate convallis. Suspendisse vitae neque in velit commodo aliquet ac eu lacus. Nunc at libero vel enim tincidunt interdum vitae congue leo. Aliquam sollicitudin venenatis ante, eget mollis ipsum tristique sed. Etiam efficitur, elit at dapibus elementum, eros risus ultrices magna, vitae consequat ligula nibh eget mauris. Etiam eu quam tellus. Vivamus porta gravida metus id finibus. Morbi ullamcorper, nisi in facilisis iaculis, mi velit consectetur odio, a blandit nisl est at mi. Vivamus sit amet magna condimentum, molestie neque tempus, ultricies ex. Morbi malesuada, massa a semper semper, sapien lorem lobortis tellus, eget dapibus mi arcu sit amet est. Aliquam luctus, velit vitae molestie luctus, quam ex ornare nisi, in blandit quam diam non velit. Aliquam id nisl vel nisi ultrices efficitur. Maecenas vitae ante non ex malesuada dapibus. Sed eu est eros.
Nunc pretium neque in nisl mollis, eget feugiat leo consectetur. Vestibulum ante ipsum primis in faucibus orci luctus et ultrices posuere cubilia Curae; Nunc pellentesque cursus orci eu mattis. Nam eget hendrerit ex, ac bibendum est. Interdum et malesuada fames ac ante ipsum primis in faucibus. Quisque faucibus ligula ac egestas pulvinar. Nam luctus, quam pulvinar tincidunt aliquam, ipsum leo dapibus eros, tincidunt scelerisque eros arcu a lectus. Maecenas eu orci vitae lacus pulvinar maximus quis vitae lacus.
Curabitur ac urna iaculis ex molestie malesuada. Nullam a facilisis arcu, ut tincidunt lectus. Cum sociis natoque penatibus et magnis dis parturient montes, nascetur ridiculus mus. In at urna aliquet, blandit nisl sit amet, suscipit lectus. Maecenas tincidunt pellentesque eros, varius laoreet massa lobortis quis. Aliquam placerat massa ac tortor iaculis, id gravida ipsum tincidunt. {\bf \Large Hasta 400 palabras}
\section{Resultados esperados}
\begin{itemize}
\item Ut non diam neque. Ut dapibus velit at mauris varius, nec volutpat nisl bibendum. Proin egestas enim et diam mattis ullamcorper.
%
\item Ut commodo enim quis auctor lobortis. Curabitur iaculis nunc tempor ornare elementum. Nullam a semper turpis, eget interdum tellus.
%
\item Class aptent taciti sociosqu ad litora torquent per conubia nostra, per inceptos himenaeos. Suspendisse consectetur mattis purus eu rhoncus. Nullam non rutrum arcu.
\end{itemize}
\section{Participantes}
\subsection{Responsable}
Nomen Nescio, area, dependencia, email
\subsection{Colaboradores} 
\begin{itemize}
\item cklndlkcvn, \'area, Dependencia, email
\item cklndlkcvn, \'area, Dependencia, email
\item cklndlkcvn, \'area, Dependencia, email
\end{itemize}

%%%%%%%%%%%%%%%%%%%
%% Aquí viene la bibliografía
%% Se etiqueta con números y la bibliografía se lista por orden de aparición en el texto
%% Se puede colocar bibliografía por capítulos. Sin embargo, en este ejemplo se hace al final del documento
%%%%%%%%%%%%%%%%%%%
\bibliographystyle{unsrt}
\bibliography{bibliografia}
\appendix
\newpage
\section{Anexos}
\begin{itemize}
\item Anexo 1 (Ej. Informe de avance)
\item Anexo 2 (Ej. Resultados parciales)
\item Anexo 3 (Ej. Recursos disponibles)
\item Anexo 4 (Ej. Cronograma de actividades)
\item Anexo 5 (Ej. Necesidades de financiamiento futuras)
\end{itemize}

\end{document}
